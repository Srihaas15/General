\let\negmedspace\undefined
\let\negthickspace\undefined
\documentclass[journal,12pt,onecolumn]{IEEEtran}
\usepackage{cite}
\usepackage{tikz}
\usepackage{circuitikz}
\usepackage{amsmath,amssymb,amsfonts,amsthm}
\usepackage{amsmath}
\usepackage{algorithmic}
\usepackage{graphicx}
\usepackage{textcomp}
\usepackage{xcolor}
\usepackage{txfonts}
\usepackage{listings}
\usepackage{multicol}
\usepackage{enumitem}
\usepackage{mathtools}
\usepackage{gensymb}
\usepackage{comment}
\usepackage[breaklinks=true]{hyperref}
\usepackage{tkz-euclide} 
\usepackage{listings}
\usepackage{gvv}                                        
\usepackage[latin1]{inputenc}                                
\usepackage{color}                                            
\usepackage{array}                                            
\usepackage{longtable}                                       
\usepackage{calc}                                             
\usepackage{multirow}                                         
\usepackage{hhline}                                           
\usepackage{ifthen}                                           
\usepackage{lscape}
\usepackage{tabularx}
\usepackage{array}
\usepackage{float}


\newtheorem{theorem}{Theorem}[section]
\newtheorem{problem}{Problem}
\newtheorem{proposition}{Proposition}[section]
\newtheorem{lemma}{Lemma}[section]
\newtheorem{corollary}[theorem]{Corollary}
\newtheorem{example}{Example}[section]
\newtheorem{definition}[problem]{Definition}
\newcommand{\BEQA}{\begin{eqnarray}}
\newcommand{\EEQA}{\end{eqnarray}}
\newcommand{\define}{\stackrel{\triangle}{=}}
\theoremstyle{remark}
\newtheorem{rem}{Remark}

\begin{document}
\bibliographystyle{IEEEtran}
\vspace{3cm}

\title{GATE 2007 EE}
\author{GUNDA SRIHAAS \\ EE24BTECH11026}
\maketitle

\renewcommand{\thefigure}{\theenumi}
\renewcommand{\thetable}{\theenumi}

\begin{enumerate}
    \item[69.]
     Which one of the following statements regarding the INT \brak{interrupt} and the BRQ  \brak{bus  request} pins in a CPU is true?
              \begin{enumerate}
              
                \item The BRQ pin is sampled after every instruction cycle ,but the INT is sampled after every machine cycle  
                \item Both INT and BRQ are sampled after every machine cycle                 
                \item The INT pin is sampled after every instruction cycle,but the BRQ is sampled after every machine cycle             
                \item Both INT and BRQ are sampled after every instruction cycle   
                
            \end{enumerate}
       
	\item [70.] A bridge circuit is shown in the figure below .Which one of the sequences given below is most suitable for balancing the bridge?
\begin{figure}[H]
\centering
\resizebox{0.6\textwidth}{!}{\begin{circuitikz}
\tikzstyle{every node}=[font=\Large]
\draw (8.25,0.5) to[R,l={ \LARGE $R_2$}] (4.75,4);
\draw (4.75,4) to[L,l={ \Large $jX_1$} ] (6.5,5.75);
\draw (6.5,5.75) to[R,l={ \LARGE $R_1$}] (8.25,7.5);
\draw (8.25,7.5) to[R,l={ \LARGE $R_3$}] (11.75,4);
\draw (11.75,4) to[R,l={ \LARGE $R_4$}] (10,2.25);
\draw (10,2.25) to[C,l={ \Large $-jX_4$}] (8.25,0.5);
\node at (8.25,5.5) [circ] {};
\node at (8.25,5.5) [circ] {};
\node at (8.25,2) [circ] {};
\draw (8.25,7.5) to[short] (8.25,5.5);
\draw (8.25,2) to[short] (8.25,0.5);
\draw (4.75,4) to[short] (2.5,4);
\draw (11.75,4) to[short] (14.25,4);
\draw (14.25,4) to[short] (14.25,-1.25);
\draw (14.25,-1.25) to[sinusoidal voltage source, sources/symbol/rotate=auto] (2.5,-1.25);
\draw (2.5,4) to[short] (2.5,-1.25);
\end{circuitikz}
}
\end{figure}

        \begin{enumerate}    
                \item First adjust $R_4$, and then adjust $R_1$
                \item First adjust $R_2$, and then adjust $R_3$
                \item First adjust $R_2$, and then adjust $R_4$
                \item First adjust $R_4$, and then adjust $R_2$
       \end{enumerate}
\section{Common Data Questions}
\textbf{Common Data for Questions $71,72,73:$} \\
A three phase squirrel cage induction motor has a starting current of seven times the full load current and full load slip of $5\%$

    \item [71.] If an autotransformer is used for reduced voltage starting to provide $1.5$ per unit starting torque,the autotransformer ratio \brak{\%} should be
        \begin{enumerate}    
        \begin{multicols}{4}
                \item $57.77\%$
                \item $72.56\%$
                \item $78.25\%$
                \item $81.33\%$
                \end{multicols}
       \end{enumerate}
    \item  [72.] If a star-delta starter is used to star this induction motor,the per unit starting torque will be
        \begin{enumerate}
        \begin{multicols}{4}
                \item $0.607$ 
                \item $0.816$ 
                \item $1.225$ 
                \item $1.616$ 
                \end{multicols}
               \end{enumerate}

    \item [73.]If a starting torque of $0.5$ per unit is required then the per unit starting current should be 
        \begin{multicols}{4}
            \begin{enumerate}
                \item $4.65$ 
                \item $3.75$ 
                \item $3.16$
                \item $2.13$
            \end{enumerate}
        \end{multicols}
\textbf{Common Data for Questions $74,75:$} \\    
   A $1:1$ Pulse Transformer (PT) is used to trigger the SCR in the below figure. The $SCR$ is rated at $1.5 kV, 250 A$ with $I_L=250 mA, I_H=150 mA, and I_Gmax=150 mA, I_Gmin=100 mA$. The SCR is connected to an inductive load, where $L = 150 mH$ in series with a small resistance and the supply voltage is $200 V$ dc. The forward drops of all transistors/diodes and gate-cathode junction during ON state are $1.0 V$ 
   
  \item [74.] The resistance $R$ should be
\begin{enumerate}
\begin{multicols}{4}
    \item $4.7 k\Omega$
    \item $470 \Omega$
    \item $47 \Omega$
    \item $4.7 \Omega$
    \end{multicols}
\end{enumerate}  
    \item [75.] The minimum approximate volt-second rating of pulse transformer suitable for triggering the $SCR$ should be : (volt-second rating is the maximum of product of the voltage and the width of the pulse that may applied)
      \begin{enumerate}    
        \begin{multicols}{4}
                \item $2000 \mu$V-s
                \item $200 \mu$V-s
                \item $20 \mu$V-s
                \item $2.0 \mu$V-s
                \end{multicols}
              \end{enumerate}   
\section{Linked Answer Questions:Q.76 to Q.85 carry two marks each.}              
 \textbf{Statement for Linked Answer Questions $76\&77:$} \\        
 An inductor designed with $400$ turns coil wound on an iron core of $16 cm^2$ cross-sectional area and with a cut of an air gap length of 1 mm. The coil is connected to a $230 V, 50 Hz$ AC supply. Neglect coil resistance, core loss, iron reluctance and leakage inductance. ($\mu_0 = 4\pi \times 10^{-7}$ H/m)
    \item [76.] The current in the inductor is
    \begin{enumerate}
    \begin{multicols}{4}
        \item $18.08 A$
        \item $9.04 A$
        \item $4.56 A$
        \item $2.28 A$
        \end{multicols}
    \end{enumerate}
    
    \item [77.] The average force on the core to reduce the air gap will be
    \begin{enumerate}
    \begin{multicols}{4}
        \item $829 N$
        \item $1666.22 N$
        \item $3332.47 N$
        \item $6664.84 N$
        \end{multicols}
    \end{enumerate}


\textbf{Statement for Linked Answer Questions 78 \& 79:}

Cayley-Hamilton Theorem states that a square matrix satisfies its own characteristic equation. Consider a matrix
\begin{align}
\myvec{ -3 & 2 \\
         -1 & 0 } 
\end{align}


    \item [78.] $A$ satisfies the relation
    \begin{enumerate}
    \begin{multicols}{4}
        \item $A^3 + 3A + 2I = 0$
        \item $A^2 + 2A + 2I = 0$
        \item $(A + I)(A + 2I) = 0$
        \item $\exp(A) = 0$
        \end{multicols}
    \end{enumerate}

    \item [79.] $A^9$ equals
    \begin{enumerate}
    \begin{multicols}{4}
        \item $511 A + 510 I$
        \item $309 A + 104 I$
        \item $154 A + 155 I$
        \item $\exp(9A)$
        \end{multicols}
    \end{enumerate}




Consider the R-L-C circuit shown in figure.

\begin{figure}[!ht]
\centering
\resizebox{0.4\textwidth}{!}{\resizebox{0.4\textwidth}{!}{%
\begin{circuitikz}
\tikzstyle{every node}=[font=\small]
\draw (3.25,9.75) to[R] (6,9.75);
\draw (6,9.75) to[L,l={ \small $L=10mH$} ] (8.5,9.75);
\draw (8.5,7) to[C,l={ \normalsize $C=10 \mu F$}] (8.5,9.75);
\draw  (3.25,8.5) circle (0.25cm);
\draw (3.25,9.75) to[short] (3.25,8.75);
\draw (3.25,8.25) to[short] (3.25,7);
\draw (3.25,7) to[short] (8.5,7);
\node [font=\footnotesize] at (4.5,10.25) {R=10 $\Omega$};
\draw [short] (8.75,9.75) -- (9.5,9.75);
\draw [short] (8.75,7) -- (9.5,7);
\draw [->, >=Stealth] (9.25,8.5) -- (9.25,9.75);
\draw [short] (9.25,8) -- (9.25,7);
\node [font=\normalsize] at (9.25,8.25) {$e_0$};
\end{circuitikz}
}%
\label{fig:my_label}
}
\end{figure}


    \item [80.] For a step-input $e_i$, the overshoot in the output $e_o$ will be
    \begin{enumerate}
        \item $0$, since the system is not under-damped
        \item $5\%$
        \item $16\%$
        \item $48\%$
   \end{enumerate}

    \item [81.] If the above step response is to be observed on a non-storage CRO, then it would be best to have the $e_i$ as 
     \begin{enumerate}    
                \item Step function
                \item Square wave of $50 Hz$
                \item Square wave of $300 Hz$
                \item Square wave of $2.0 KHz$
        \end{enumerate} 
     
      
  \textbf{Statement for Linked Answer Questions 82 \& 83:}

The associated figure shows the two types of rotate right instructions R1, R2 available in a microprocessor where \textit{Reg} is an 8-bit register and \textit{C} is the carry bit. The rotate left instructions L1 and L2 are similar except that \textit{C} now links the most significant bit of \textit{Reg} instead of the least significant one.
\begin{figure}[H]
\centering
\resizebox{0.6\textwidth}{!}{\begin{circuitikz}
\tikzstyle{every node}=[font=\Large]
\draw  (5.25,12) rectangle (9.5,11);
\draw [->, >=Stealth] (4.5,11.5) -- (5.25,11.5);
\draw (4.5,11.5) to[short] (4.5,10.5);
\draw [->, >=Stealth] (9.5,11.5) -- (10.25,11.5);
\draw  (10.25,12) rectangle (11.25,11);
\draw (11.25,11.5) to[short] (12,11.5);
\draw (12,11.5) to[short] (12,10.5);
\draw (12,10.5) to[short] (4.5,10.5);
\draw  (5.25,9.75) rectangle (9.5,8.75);
\draw [->, >=Stealth] (4.5,9.25) -- (5.25,9.25);
\draw (4.5,9.25) to[short] (4.5,8.25);
\draw [->, >=Stealth] (9.5,9.25) -- (10.25,9.25);
\draw  (10.25,9.75) rectangle (11.25,8.75);
\draw (9.75,9.25) to[short] (9.75,8.25);
\draw (9.75,8.25) to[short] (4.5,8.25);
\node [font=\Large] at (7.25,11.5) {Reg};
\node [font=\Large] at (10.75,11.5) {C};
\node [font=\Large] at (7.25,9.25) {Reg};
\node [font=\Large] at (10.75,9.25) {C};
\node [font=\Large] at (3.5,11.25) {$R_1:$};
\node [font=\Large] at (3.5,8.75) {$R_2:$};
\end{circuitikz}
}
\end{figure}



    \item [82.] Suppose \textit{Reg} contains the $2's complement number 11010110$. If this number is divided by 2 the answer should be
    \begin{enumerate}
        \item $01101011$
        \item $10010101$
        \item $11110001$
        \item $11101011$
    \end{enumerate}
    \item [83.] Such a division can be correctly performed by the following set of operations
    \begin{enumerate}[label=(\Alph*)]
        \item L2, L2, R1
        \item L2, R1, R2
        \item R2, L1, R1
        \item R1, L2, R2
    \end{enumerate}


\textbf{Statement for Linked Answer Questions 84 \& 85:}


\item [84.] A signal is processed by a causal filter with transfer function $G(s)$. For a distortion-free output signal waveform, $G(s)$ must
    \begin{enumerate}
        \item provide zero phase shift for all frequency
        \item provide constant phase shift for all frequency
        \item provide linear phase shift that is proportional to frequency
        \item provide a phase shift that is inversely proportional to frequency
    \end{enumerate}

    \item [85.] $G(z) = \alpha z^2 + \beta z^5$ is a low-pass digital filter with a phase characteristic same as that of the above question if
    \begin{enumerate}
        \item $\alpha = \beta$
        \item $\alpha = -\beta$
        \item $\alpha = \beta^{1/3}$
        \item $\alpha = \beta^{-(1/3)}$
    \end{enumerate}

    
\end{enumerate}
\end{document}





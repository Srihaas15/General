\let\negmedspace\undefined
\let\negthickspace\undefined
\documentclass[journal,12pt,twocolumn]{IEEEtran}
\usepackage{cite}
\usepackage{amsmath,amssymb,amsfonts,amsthm}
\usepackage{algorithmic}
\usepackage{graphicx}
\usepackage{textcomp}
\usepackage{xcolor}
\usepackage{txfonts}
\usepackage{listings}
\usepackage{enumitem}
\usepackage{mathtools}
\usepackage{gensymb}
\usepackage{comment}
\usepackage[breaklinks=true]{hyperref}
\usepackage{tkz-euclide} 
\usepackage{listings}
\usepackage{gvv}  
\usepackage{tikz}
\usepackage{circuitikz} 
\usepackage{caption}
\def\inputGnumericTable{}              
\usepackage[latin1]{inputenc}          
\usepackage{color}                    
\usepackage{array}                     
\usepackage{longtable}                 
\usepackage{calc}                     \usepackage{multirow}                  
\usepackage{hhline}                    
\usepackage{ifthen}                    
\usepackage{lscape}
\usepackage{amsmath}
\newtheorem{theorem}{Theorem}[section]
\newtheorem{problem}{Problem}
\newtheorem{proposition}{Proposition}[section]
\newtheorem{lemma}{Lemma}[section]
\newtheorem{corollary}[theorem]{Corollary}
\newtheorem{example}{Example}[section]
\newtheorem{definition}[problem]{Definition}
\newcommand{\BEQA}{\begin{eqnarray}}
\newcommand{\EEQA}{\end{eqnarray}}
\newcommand{\define}{\stackrel{\triangle}{=}}
\theoremstyle{remark}
\newtheorem{rem}{Remark}

%\bibliographystyle{ieeetr}
\begin{document}
%

\bibliographystyle{IEEEtran}




\title{
%	\logo
CIRCLES

\large{EE1030 : MATRIX THEORY}

Indian Institute of Technology Hyderabad
%	}

\author{Srihaas Gunda}

(EE24BTECH11026)
}





\maketitle

\newpage



\bigskip

\renewcommand{\thefigure}{\theenumi}
\renewcommand{\thetable}{\theenumi}

\section{JEE Advanced/IIT-JEE}
\subsection{Comprehension Based Question}
\subsubsection{Passage }
1\\
ABCD is a square of side length $2$ units.$C_1$ is the circle touching all the sides of the square ABCD and $C_2$ is the $circumcircle$ of square ABCD.L is a fixed line in same plane and R is a fixed point.\\
\begin{enumerate}
\item If P is any point of $C_1$ and Q is another point on $C_2$,then $\displaystyle\frac{PA^2+PB^2+PC^2+PD^2}{QA^2+QB^2+QC^2+QD^2}$\\
\hfill ($2006$-$5M$,-$2$)\\
\begin{enumerate}
\item 075\\
\item 1.25\\
\item 1\\
\item 0.5\\
\end{enumerate}
\item If a circle is such that it touches the line L and the circle $C_1$ externally,such that both the circles are on the same side of the line,then locus of centre of the circle 
\hfill($2006$-$5M$,-$2$)
\begin{enumerate}
\item ellipse\\
\item hyperbola\\
\item parabola\\
\item circle\\
\end{enumerate}
\item A line L'through A is drawn parallel to BD.Point S moves such that its distances from the line BD and the vertex A are equal.If locus of S cuts L' at $T_2$ and $T_3$ and AC at $T_1$,then area of $\Delta T_1T_2T_3$ is\\
\hfill($2006$-$5M$,-$2$)\\
\begin{enumerate}
\item $1/2$ sq.units\\
\item $2/3$ sq.units\\
\item $1$ sq.units\\
\item $2$ sq.units\\
\end{enumerate}
\end{enumerate}
\subsubsection{Passage } 2\\
A circle C of radius $1$ unit is inscribed in an equilateral triangle PQR.The points of contact of C with sides PQ,QR,RP are D,E,F respectively.The line PQ is given by the equation $\sqrt{3}x+y-6=0$ and the point D is ($3\sqrt{3}/2, 3/2$).Further,it is given that the origin and the centre of C are on same side of line PQ.\\
\\
4.The equation of circle C is\hfill{(2008)}
\begin{enumerate}
\item $(x-2\sqrt{3})^2 + (y-1)^2=1$\\
\item $(x-2\sqrt{3})^2 + (y+1/2)^2=1$\\
\item $(x-\sqrt{3})^2 + (y-1)^2=1$\\
\item $(x-\sqrt{3})^2 + (y+1)^2=1$\\
\end{enumerate}
\subsection{Assertion \& Reason Type Questions}
\begin{enumerate}
\item Tangents are drawn from point ($17,7$) to the circle $x^2+y^2=169$.\\
STATEMENT-1:The tangents are mutually perpendicular.because\\
STATEMENT-2:The locus of all points from which mutually perpendicular tangents can be drawn to a given circle is $x^2+y^2=338.$ \hfill{(2007-3M)}\\
\begin{enumerate}
\item Statement-1 is True,statement-2 is True;Statement-2 is a correct explantion for Statement-1.\\
\item Statement-1 is True,statement-2 is True;Statement-2 is NOT a correct explantion for Statement-1.\\
\item Statement-1 is True,Statement-2 is False\\
\item Statement-1 is False,Statement-2 is True.\\
\end{enumerate}

\item Consider $L_1:2x+3y+p-3=0$\\
               $L_2:2x+3y+p+3=0$\\
where p is a real number,and C: $x^2+y^2+6x-10y+30=0$\\
STATEMENT-1:If line $L_1$ is a chord of circle C ,then line $L_2$ is not always a diameter of circle C\\
and\\
STATEMENT-2:Ifline $L_1$ is a diameter of circle C ,then line $L_2$ is not a chord  of circle C. \hfill{(2008)}\\
\begin{enumerate}
\item Statement-1 is True,statement-2 is True;Statement-2 is a correct explantion for Statement-1.\\
\item Statement-1 is True,statement-2 is True;Statement-2 is NOT a correct explantion for Statement-1.\\
\item Statement-1 is True,Statement-2 is False\\
\item Statement-1 is False,Statement-2 is True.\\
\end{enumerate}
\end{enumerate}
\subsection{Integer Value Correct Type}
\begin{enumerate}
\item The centres of two circles $C_1 and C_2$ each of unit radius are at a distance of 6 units from each other. Let P be the midpoint of the line segment joining the centres of $C_1 AND C_2$ and C be a circle touching circles $C_1 AND C_2$ externally.If a common tangent to $C_1$ and C passing through P is also a common tangent to $C_2$ and C, then the radius of circle C is \hfill{(2009)}\\
\item The straight line $2x-3y=1$ divides the circular region $x^2+y^2\leq6$ into two parts.\\
If S is {$(2,3/4),(5/2,3/4),(1/4,-1/4),(1/8,1/4)$} then the  number of point(s) in S lying inside the smaller part is \hfill{(2011)}\\
\\
\item For how many values of p, the circle $x^2+y^2+2x+4y-p=0$ and the coordinate axes have exactly three common points? \hfill{(JEE Adv. 2017)}\\
\\
\item Let the point B be the reflection of the point A(2,3) with respect to the line $8x-6y-23=0$.Let $T_A$ and $T_B$ be circles of radii $2$ and $1$ with centres A and B respectively.Let T be a common tangent to the circles $T_A$ and $T_B$ such that both the circles are on the same side of T.If C is the point of intersection of T and the line passing through A and B,then the length of the line segment AC is \hfill{(JEE Adv. 2019)}
\end{enumerate}
\section{JEE Main / AIEEE}
\begin{enumerate} 
\item If the chord $y=mx+1$ of the circle $x^2+y^2=1$ subtends an angle of measure $45^0$ at the major segment of the circle then the value of m is \hfill{(2002)}\\
\begin{enumerate}
\item$2\pm\sqrt{2}$\\
\item$-2\pm\sqrt{2}$\\
\item$-1\pm\sqrt{2}$\\
\item none of this\\
\end{enumerate}
\item The centres of a set of circles,each of radius $3$, lie on the circle $x^2+y^2=25$.The locus of any point in the set is \hfill{(2002)}\\
\begin{enumerate}
\item$4$ $\leq$$x^2+y^2$ $\leq$ $64$\\
\item$x^2+y^2\leq25$\\
\item$x^2+y^2\geq25$\\
\item$3$ $\leq$ $x^2+y^2$ $\leq$ $9$\\
\end{enumerate}
\item The centre of the circle passing through(0,0) and (1,0) and touching the circle $x^2+y^2=9$ is \hfill{(2002)}\\
\begin{enumerate}
\item$(1/2,1/2)$\\
\item$(1/2,-\sqrt{2})$\\
\item$(3/2,1/2)$\\
\item$(1/2,3/2)$\\
\end{enumerate}
\item The equation of a circle with origin as a centre and passing through equilateral triangle whose median is of length 3a is \hfill{(2002)}\\
\begin{enumerate}
\item$x^2+y^2=9a^2$\\
\item$x^2+y^2=16a^2$\\
\item$x^2+y^2=4a^2$\\
\item$x^2+y^2=a^2$\\
\end{enumerate}
\item If the two circles $(x-1)^2+(y-3)^2=r^2$ and $x^2+y^2-8x+2y+8=0$ intersect in two distinct points, then \hfill{(2003)}\\
\begin{enumerate}
\item$r>2$\\
\item$2<r<8$\\
\item$r<2$\\
\item$r=2$
\end{enumerate}
\end{enumerate}
\end{document}
